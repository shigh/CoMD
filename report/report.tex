\documentclass[12pt]{article}
\usepackage[margin=1.0in]{geometry}

%\usepackage[utf8]{inputenc}
%\usepackage[T1]{fontenc}
%\usepackage{fixltx2e}
%\usepackage{graphicx}
%\usepackage{longtable}
%\usepackage{float}
%\usepackage{wrapfig}
%\usepackage{rotating}
%\usepackage[normalem]{ulem}
%\usepackage{textcomp}
%\usepackage{marvosym}
%\usepackage{wasysym}
%\tolerance=1000

\usepackage{amsmath}
\usepackage{amssymb}
\usepackage{amsfonts}
\usepackage{bm}
\usepackage{algpseudocode}
\usepackage{hyperref}

\usepackage{listings}

\newcommand{\pd}[2]{\frac{\partial #1}{ \partial #2}}
\renewcommand{\v}[1]{\bold{#1}}

\title{CS 598 Final Project}
\author{Scott High and Erin Molloy}
\date{\today}


\begin{document}

\maketitle

\section{Introduction}
Molecular dynamics (MD) simulations are important for researching physical phenomena 
such as, the electronic structure of metal and the folding trajectories of proteins. 
In our final project, we will create performance expectations for a molecular dynamics 
simulation mini-app, called \href{https://github.com/exmatex/CoMD}{\texttt{CoMD}}, which is designed to expose the core features 
of molecular dynamics simulations in a code base that is simple to understand and modify.

% Outline contributions

%In our final project, we will create performance expectations for the Lennart-Jones(LJ) model 
%of potential energy. We propose to
%\begin{itemize}
%    \item [1] Create a performance expectation based on memory access patterns.
%    \item [2] Create a performance expectation of scalability based on communication.
%    \item [3] Attempt to improve performance and scalability.
%    \item [4] Compare the scalability of the Lennart-Jones model to that of the Embedded Atom Model.
%\end{itemize}

\subsection{Background}
Molecular dynamics simulations model the movement of individual particles over time. 
Particle motion is determined by Newton's well-known equation of motion, $F = ma$. 
In an $N$-particle simulation, the force on the particle at position 
$\bm{r}_i = (x_i, y_i, z_i)$ is given by
\begin{equation*}
    F_i = m_i \ddot{\bm{r}}_i = -\frac{\partial}{\partial \bm{r}_i} U(\bm{r}_1, \dots, \bm{r}_N)
\end{equation*}
The system of first order ODEs is integrated to find the position of particles at each time step.
Both the LJ and EAM models utilize inter-particle interactions or ``pair potentials''
within a predefined region of interaction.


% Compilation requires only a valid \texttt{C} compiler and a working \texttt{MPI} implementation.
% The code is relatively short, only a few thousand lines, and is extensively documented. 

% \subsection{Memory access}
% Particles are stored in one-dimensional arrays and accessed via mapping functions
% [\href{https://github.com/exmatex/CoMD/blob/master/src-mpi/linkCells.c}{\texttt{linkCells.c}}, 
% $\sim$450 lines]. We will consider the two compute kernels 
% for calculating inter-particle potentials
% [\href{https://github.com/exmatex/CoMD/blob/master/src-mpi/ljForce.c}{\texttt{ljForce.c}},
% $\sim$200 lines;
% \href{https://github.com/exmatex/CoMD/blob/master/src-mpi/eam.c}{\texttt{eam.c}},
% $\sim$800 lines].


% \subsection{Communication}
% Particles are decomposed into three-dimensional subdomains 
% [\href{https://github.com/exmatex/CoMD/blob/master/src-mpi/decomposition.c}{\texttt{decomposition.c}},
% $\sim$60 lines]. 
% We will consider the data transferred between subdomains 
% [\href{https://github.com/exmatex/CoMD/blob/master/src-mpi/haloExchange.c}{\texttt{haloExchange.c}},
% $\sim$600 lines] and the specific calls to \texttt{MPI}
% [\href{https://github.com/exmatex/CoMD/blob/master/src-mpi/parallel.c}{\texttt{parallel.c}},
% $\sim$200 lines].

\section{Performance Baseline}
% Erin Molloy

Detailed code timings are already built into \texttt{CoMD}. 



%
% 
%
%We will run performance
%experiments with the provided model of copper atoms in a lattice.

%
% 
%
\subsection{Single processor}
A performance baseline is obtained by running \texttt{CoMD} on a single processor.
and the size of the lattice by a factor of 10 in each dimension.


We will examine problem sizes larger than the size of the L2 cache (2MB - 
\href{https://bluewaters.ncsa.illinois.edu/user-guide}{\texttt{Blue Waters User Guide}}).

\subsection{Strong scaling}
% add graphs

\subsection{Weak scaling}
%A performance baseline will be obtained  running \texttt{CoMD} on multiple processors in
%both strong an weak scaling studies. The \texttt{CoMD} code repository includes information
%and scripts for performing these: 
%\url{http://exmatex.github.io/CoMD/doxygen-mpi/pg_running_comd.html}.
%In selecting the problem sizes, we will need to consider the number of nodes 
%each with 64GB of memory and two processors. 

% The network topology of Blue Waters is a three-dimensional torus, so it is also theoretically possible to select 
% an allocation such that subdomains communicate efficiently with neighbors 
% (\href{https://bluewaters.ncsa.illinois.edu/topology-aware-scheduling}{Blue Waters Topology
% Aware Scheduling}). In particular, we should try to insure that both the performance 
% baseline and other experiments are allocated to sets of nodes with similar topologies.
%two compute nodes on 
%each gemini hub enabling neighbor-to-neighbor communication without network contention.


\section{Performance model}
All analysis is done assuming a one-level cache model and
approximating read and write times as equal.  The constants used in
the performance expectations are determined by the clock rate and
\texttt{STREAM} benchmark results as measured on Blue Waters.

The equations we are calculating for each particle in computer variables are
\begin{align}
  E_{tot} &= \sum_{ij} U_{LJ}(r_{ij}) \\
  U_{LJ}(r_{ij}) &= 
  A\left(\frac{1}{r_{ij}}\right)^{6}\left\{ \left(\frac{1}{r_{ij}}\right)^{6} - 1 \right\} \\
  \notag\\
  \textbf{F}(r_{ij}) &= - U'_{LJ}(r_{ij})\hat{r}_{ij} \notag\\
      &=  24 \frac{\epsilon}{r_{ij}} \left\{ 2 \left(\frac{\sigma}{r_{ij}}\right)^{12}
              - \left(\frac{\sigma}{r_{ij}}\right)^6 \right\} \hat{r}_{ij} \notag\\
              &=  A \frac{1}{r_{ij}^2} \left(\frac{1}{r_{ij}}\right)^{6} \left\{ 2 \left(\frac{1}{r_{ij}}\right)^{6}
              - 1 \right\} \textbf{r}_{ij}
\end{align}
in a fully periodic domain. The total number of simulation particles
is $N$, and each particle has $n$ neighbors within
$r_c=r_{\text{cutoff}}$.

We develope our performance model by calculate the cost of updating
particle $i$ and its interaction with each of its $n$ neighbors
$j$. The cost is:
\begin{itemize}
\item[A)] 3 loads for particle $i$'s position
\item[B)] For each particle $j$ where $r_{ij}<r_c$
  \begin{enumerate}
    \item $3$ loads for particle $j$'s position
    \item 3 subtractions, 3 multiplications, 2 additions and 1 division to calculate $\frac{1}{r_{ij}^2}$ 
    \item 3 multiplications to calculate $\left(\frac{1}{r_{ij}^2}\right)^3=\left(\frac{1}{r_{ij}}\right)^6$
    \item 2 subtractions and 2 multiplications to calculate $U_{LJ}$ (1 subtraction is for cutoff potential)
    \item 3 multiplications and 1 subtraction to calclulate $F(r_{ij})$
    \item 3 multiplications and 3 additions to calculate and update $F_{x,y,z}$
  \end{enumerate}
\item[C)] 1 write to save $U_{LJ}$. (Assume we can save the $U_{LJ}(r_{ji})$ term for free)
\item[D)] 3 writes to save $F_{x,y,z}$
\item[E)] 1 addition to update $E$
\end{itemize}
The total cost is $N(A+nB/2)+C+D+E$, where the $1/2$ is to account for
double counting. Using $c$ for the cost of a floating point operation,
$w$ for the cost of a write and $r$ for the cost of a read the
estimated cost is
\begin{equation}
N \left(\frac{n}{2} \left(26 c + 3 w\right) + 3 w\right) + c + 4 r
\end{equation}
Approximating reads and writes as equal reduces this to
\begin{equation}
  N \left(\frac{n}{2} \left(26 c + 3 r\right) + 3 r\right) + c + 4 r
\end{equation}



The data structure used for storing the computational particles is a linear C array.
A link cell is created to efficiently identify cells within the particle interaction distance 

% include picture of particle lattice...

% Graphs -- Erin


\section{Modifications}
% Scott

% -- compiler choice
% -- noticing that in either choice there is no vectorization
% -- dereference pointers to arrays before 4 loops
% -- still no vectorization, why?, why hard to fix?

% -- linkcells, ordering is already optimal, notes on double counting instead of search


\section{Conclusion}
% REFERENCES???

% Implementing a new data structure is beyond the scope of 
% this project, but simple changes to improve access patterns, such as changing the order of particles 
% [within linked cells] in the one-dimensional array, is feasible. We will also examine the loops in the compute 
% kernel for potential modifications, such as, loop unrolling. Vectorization and pipelining could also be added.

% \subsection{Communication}
% We will use the scalability expectations to measure the effectiveness of distributed memory parallelism 
% in the Lennart-Jones potential solver. Halos are exchanged using the function \texttt{sendReceiveParallel()}, which
% executes a blocking send/receive operation via \texttt{MPI\_Sendrecv}. Potential modifications
% include implementing non-blocking communication using \texttt{MPI\_Isend}, \texttt{MPI\_Irecv}, and
% \texttt{MPI\_Waitall} (lecture 25 slide 32). This appears to be effective for messages sizes over 10 bytes. 
% Thus, we will also need to examine the size of the halos.

% \section{Comparing Results}
% \label{sec-6}
% The improved code will be run using the same model problem and problem
% sizes as the baseline measurement. As mentioned above the \texttt{CoMD}
% mini-app already has implemented detailed timing measurements. We do
% not intend to change the overall structure of the code, so the built in measurements 
% should be sufficient for these comparing run times.

\end{document}
